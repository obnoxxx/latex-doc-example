\documentclass{article}
\usepackage{amsmath}

\title{example \LaTeX\ article}
\author{obnox (Michael Adam)}

\usepackage{listings}
\lstset{breaklines=true}
%
\usepackage{imakeidx}
% generate an index file:
\makeindex


\begin{document}
\maketitle

\begin{abstract}

    This is a simple  example article document to get started with  {\LaTeX}.

\end{abstract}

\addcontentsline{toc}{section}{Contents}
\tableofcontents

\addcontentsline{toc}{section}{Introduction}
\section*{Introduction}

This is an example document written in {\LaTeX}, Leslie Lamport's\index{Lamport, Leslie} typesetting
sysyem which is based on Donald E. Knuth's\index{Knuth, Donald E.} {\TeX}.

The main philosphy behind {\LaTeX}---in contrast to most word processors\index{word processor} --- is to allow and encourage the author to
focus on the content of the text rather than on the layout\index{layout}and formatting.

writing documents in {\LaTeX} is a bit like programming\index{programming}: One writes a ``source code''\index{source code}  file with the content in plain text
interspersed with some {\LaTeX} commands\index{commands}
and then ``compiles'' or processes it with the processorss \verb|latex| or \verb|pdflatex| to produce
putput files in formats like \textbf{PDF}\index{pdf} or {\TeX}' own device independent format \textbf{DVI}\index{dvi}.


\paragraph{}some more introductory words.
And this is already the end of the introduction.




\section{Beginnings}
\label{sec:beginnings}

This first section explains and demonstrates some concepts of {\LaTeX}.

Paragraphs are automatically created from blocks of text separated by one or more
blank lines.


\paragraph{} Some more text goes here in a new paragraph.
Stil confused how the paragraph separation really works?
New paragraphs can also be started explicitly with the \verb|\paragraph{}| command.



\paragraph{} Section headers and other structure elements are automatically formatted and numered
if one uses the corresponding commands like \verb|\section|, \verb|\chapter|, and
so on.


\paragraph{} As {\LaTeX} is based on Donald E. Knuth's\index{Knuth} {\TeX},
Writing math\index{math} formulae\index{formulae} is rather easy. in the \verb+equation+ environment.

Here is an example:


\begin{equation}
\label{eq:example}
\sum_{i=1}^{n}i = \frac{n(n+1)}{2}
\end{equation}

\paragraph{} Alternatively, formulae can be written inline, surrounded
by \verb|$| characters. for example $a^2 + b^2 = c^2$.

\section{Continuation}

\paragraph{}One big strength of {\LaTeX} is the ability to automatically create a table of contents from the
structure elements. Another is the ability to reference\index{reference}  sections and other numerable elements like formulae
and images in the text. Example: Section \textbf{Beginnings} has number \ref{sec:beginnings}.
Equation \ref{eq:example} is the numbered example equation.


\section{Indexes}

Another extremely useful feature\index{feature} of  {\LaTeX} ---especially for longer documents like books\index{books}--- is the ability to automate\index{automate} the creation of indexes\index{indexes} of keywords.

Using the package \verb+makeidx+ or \verb+imakeidx+ and placing the  command  \verb+\makeindex+ in the document''s preamble, on can use the command \verb+\index{keyword}+ in the text to add the keyword\index{keyword}  \texttt{keyword} to the index.

\verb+\index+ can be used anywhere in the text but will typically be used near an  appearance of the keyword.

When creating an index,  The comman \verb+makeindex+ has to be run between runs of \verb+latex+ an \verb+pdflatex+, respectively.

I. E. with an input file named \verb+doc.tex+, a typical compile\index{compile} sequence could look like this:


%%\begin{verbatim}

\begin{lstlisting}
 pdflatex doc
 pdflatex doc
 makeindex doc
 pdflatex doc
\end{lstlisting}
%%\end {verbatim}








\section{Conclusion}

and with this, our little example document already comes to an end.



%%\section*{Index}
% load and print  the index:
\printindex
\addcontentsline{toc}{section}{Index}


\end{document}
